% NSF Personal Statement
% Maximum 3 Pages
% ---

% ---
% Preamble
% ---
\documentclass[12pt]{article}
\usepackage[margin=1in]{geometry}


\begin{document}

% ---
% Background
% --- 

\textbf{Background:}

\bigskip

% ---
% Hypothesis
% ---

\textbf{Hypothesis:} The current optimal control tools can benefit from
distributed computing, however, this is not a capability of current optimal
control tools. By developing an optimal control tool that takes advantage of
distributed computing the NLP solver will be able to operate much faster that
current tools. This will help enable online optimal control in addition to
making the use of optimal control increasingly available for different
audiences.

Improved optimal control tools will allow greater
increases to be made with renewable energy. Can improve optimal control tools
by creating an NLP solver that can take advantage of parallel processing.

\bigskip

% ---
% Work Plan
% ---

\textbf{Work Plan:} I will begin my work by first gaining increased capability
with current NLP solvers. I will then begin writing my own NLP solver. Once the
solver has been written I work on making it capable of taking advantage of
parallel processing.  The final step will be to optimize the code for speed as
much as possible such that it will be the optimal control solver of choice.

\bigskip

% ---
% Intelectual Merit
% ---

\textbf{Intelectual Merit:} 

Distributed computing will allow large decreases in
time to compute. Will allow potential for onboard optimal control. Will allow
for solving of bigger problems. Current NLP solvers not designed to take
advantage of parallel processes

\bigskip

% ---
% Broader Impact
% ---

\textbf{Broader Impact:} The optimal control tools developed will enable new
systems to be subject to optimal control that would otherwise have been
untractable. This optimal control tool will bring a direct impact towards the
energy sustainability of the nation. In order to move towards complete energy
self sufficiency there will be many products and procedures that will need to
be done in an optimal manner, otherwise they will not have the economic
viablility to be a long term solution. An example of a place optimal control
can be used in renewable energy is the case of wind turbines that are
essentially placed in blimps so that their vertical location can be adjusted to
meet the highest wind speeds. Optimal control can be used in this location in
order to balance moving the turbine to produce maximum electricity while
expending as little energy as possible in the adjustments.

    Another place that will benefit from optimal control tools utilizing
distributed computing is space systems. On space systems the processors
have to be radiation shielded for protection but as a result they can not
be run very fast due to risk of overheating. This means in order to
increase processing speed additional processors are needed to be used. As a
consequence if optimal control is to be run fast enough to be used as online
optimal control then it will need to take advantage of multiple processors.

Increased ability to solve optimal control problems.
Fits nicely with naturally distributed problems (ex. electric grid, air traffic
control, etc.). Space processors have to be slow so that they do not overheat
inside their radiation shields and so optimal control will be feasible onboard
if multiple processors are able to be used.

\end{document}
